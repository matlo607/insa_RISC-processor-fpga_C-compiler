Au final, ce projet nous a donné une vue globale mais aussi précise sur toutes les étapes qui vont d’un programme source en ASCII à son exécution au niveau matériel. Cela nous a également permis de relier entre eux bon nombres d’enseignements des années précédentes et donc d’éclaircir certaines zones d’ombres. Le projet semblait au départ très ambitieux et en même temps très motivant. Nous aurions aimé avoir plus de temps pour implémenter d’autres fonctionnalités manquantes du langage C ainsi que certaines instructions matérielles comme \texttt{POP}, \texttt{PUSH}, \texttt{AND}, \texttt{XOR}, \ldots{} qui nous auraient facilité la traduction du langage C vers la langage assembleur.
