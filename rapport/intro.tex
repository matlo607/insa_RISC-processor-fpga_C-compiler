Le projet s'est articulé en deux axes, le premier autour de la conception d'un microprocesseur de type RISC avec pipeline, le deuxième autour du développement d'un compilateur en utilisant les logiciels LEX et YACC. L'objectif était de réaliser un système informatique complet.\\
Dans un premier temps, nous parlerons de l'architecture matérielle du microprocesseur que nous avons implémenté sur la \nexys{}, une carte de chez Xilinx possédant un FPGA Spartan6. Au cours de cette partie, nous aborderons les spécificités de notre architecture.\\
Par la suite, nous présenterons les éléments du langage C supportés par le compilateur, le fonctionnement des deux sous-modules dont il est composé et les fichiers intermédiaires générés.


